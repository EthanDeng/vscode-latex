% !TEX options=--shell-escape
%!TEX program = xelatex
\documentclass{article}
\usepackage{minted}
\setminted[latex]{fontsize=\footnotesize}
\setminted[json]{fontsize=\footnotesize}
\setminted[shell]{fontsize=\small}
\usepackage{graphicx}
\graphicspath{{image/}{figure/}{fig/}{img/}}
\usepackage[a4paper,margin=2.5cm]{geometry}
\usepackage[dvipsnames]{xcolor}
\usepackage{hyperref}
\hypersetup{
  colorlinks,
  citecolor=Violet,
  linkcolor=Red,
  urlcolor=NavyBlue}

\usepackage{xeCJK}
\usepackage{fontspec,xunicode,xltxtra}
\setmainfont{Minion Pro}
\setsansfont{Myriad Pro}
\setmonofont{Ubuntu Mono}
\setCJKmainfont[ItalicFont={方正楷体简体}, BoldFont={方正黑体简体}]{方正书宋简体}
\setCJKsansfont[BoldFont={方正黑体简体}]{方正楷体简体}
\setCJKmonofont{方正楷体简体}

\usepackage{indentfirst}
\setlength{\parindent}{2em}

\usemintedstyle{elegant}
\linespread{1.3}
\usepackage{etoolbox,xpatch}

\makeatletter
\AtBeginEnvironment{minted}{\dontdofcolorbox}
\def\dontdofcolorbox{\renewcommand\fcolorbox[4][]{##4}}
\xpatchcmd{\inputminted}{\minted@fvset}{\minted@fvset\dontdofcolorbox}{}{}
\makeatother

% caption settings 
\RequirePackage[font=small,labelfont={bf}]{caption} 
\captionsetup[table]{skip=3pt}
\captionsetup[figure]{skip=3pt}

% list/itemize/enumerate setting
\RequirePackage[shortlabels]{enumitem}
\setlist{nolistsep}

\renewcommand{\figurename}{\bfseries 图}
\renewcommand{\tablename}{\bfseries 表}

\title{\bfseries \LaTeX{} 编译环境配置:Visual Studio Code 配置简介}
\author{\href{https://ddswhu.me/}{Dongsheng Deng}}
\date{February 2, 2019}

\begin{document}
\maketitle

本文介绍如何配置 Visual Studio Code 作为 \LaTeX{} 的编辑器。Github 地址:\href{https://github.com/EthanDeng/vscode-latex}{vscode-latex},欢迎提交 issues 和 pull requests。

\section{为什么用 Visual Studio Code}
Visual Studio Code(以下简称 VS Code) 是微软推出的一个编辑器,它的优点你可以自行百度,这里不赘述。对我来说,它最有吸引力的当属在 Windows 系统,它对于中英文字体的渲染。如果你原来用过其他编辑器,你就知道在普通屏幕上,中英文的显示效果简直是灾难。我原来因为编辑器的中文显示(当然还有 Terminal 的吸引力)一度想买 Mac,当然最后因为对性能和颜值的追求并不匹配我的财力,加上 Windows 上有些软件不能舍弃,最后作罢。

\textbf{注}:高分屏加上合适的字体,Sublime Text 的显示效果也非常好。

\begin{figure}[htbp]
  \centering
  \includegraphics[width=0.9\textwidth]{vscode.png}
  \caption{Visual Studio Code 的界面图}
  \label{fig:vscode}
\end{figure}



\section{准备工作}
首先,为了搭建 \LaTeX{} 工作环境,你需要安装:

\begin{itemize}
  \item \TeX{} Live 或者 MiKTeX (本文以 \TeX{} Live 2018 为例)
  \item Visual Studio Code
  \item \LaTeX{} Workshop (VS Code 插件)
  \item SumatraPDF 阅读器(可选,用于预览 PDF)
\end{itemize}

在上述软件/插件安装之后,你需要把 \TeX{} Live 的 bin 目录(\mintinline[breaklines]{shell}{D:/Program Files/texlive/2018/bin/win32}) 以及 SumatraPDF 的路径(\mintinline[breaklines]{shell}{C:/Program Files (x86)/SumatraPDF})添加到系统环境变量(\mintinline{shell}{Path})中。


\subsection{安装插件}
VS Code 中插件安装方法如下:在左侧点击扩展按钮(Key:\mintinline{shell}{Ctrl+Shift+X}),然后搜索插件名字 \mintinline{shell}{LaTeX Workshop},选择安装即可。

\subsection{添加环境变量}
Win10 中将路径添加到环境变量中的步骤如下:右键我的电脑,然后选择 \mintinline{shell}{属性},在左侧选择 \mintinline{shell}{高级系统设置},然后选择下方的 \mintinline{shell}{环境变量},选择变量 \mintinline{shell}{Path} 编辑,将需要添加的路径添加进去即可。

\section{配置编译方式与编译组合}
VS Code 在 2018 年经历了一次大改之后,配置比原来简单了。它们把过去的 \mintinline{shell}{tool.chain} 改为了 \mintinline{shell}{recipe},其实本质上是一样的。

\subsection{编译方式(\mintinline{shell}{tool})}
VS Code 默认添加了 3 个编译工具(tools):分别是 \mintinline{shell}{latexmk},\mintinline{shell}{pdflatex} 和 \mintinline{shell}{bibtex}(所有的工具只编译一次)。编译 \mintinline{shell}{tex} 文档方法,使用右键,选择 \mintinline{shell}{Build LaTeX Project}(快捷键:\mintinline{shell}{Ctrl+Alt+B}),默认使用 \mintinline{shell}{latexmk},查看 PDF 文件使用快捷键:\mintinline{shell}{Ctrl+Alt+V}。

为了添加其他的编译方式(比如 \mintinline{shell}{xelatex}),我们需要修改 \LaTeX{} Workshop 的配置。打开 LaTeX Workshop 配置的方法如下:在 VS Code 界面左下角,点击齿轮按钮 \includegraphics[width=0.018\textwidth]{setting.png},选择\mintinline{shell}{设置},然后在设置搜索框内输入 \mintinline{shell}{latex},在搜索结果中,点击 \underline{\mintinline{shell}{在 settings.json 中编辑}},示例如图~\ref{fig:settings}:

\begin{figure}[!htbp]
  \centering
  \includegraphics[width=0.7\textwidth]{settings.png}
  \caption{打开设置文件\label{fig:settings}}
\end{figure}


打开配置文件之后,在右侧 \mintinline{shell}{用户设置} 粘贴下面 JSON 片段:

\begin{minted}[frame=single]{json}
"latex-workshop.latex.tools": [
  {
    "name": "xelatex",
    "command": "xelatex",
    "args": [
      "-synctex=1",
      "-interaction=nonstopmode",
      "-file-line-error",
      "%DOC%"
    ]
  },
  {
    "name": "xelatex-with-shell-escape",
    "command": "xelatex",
    "args": [
      "--shell-escape",
      "-synctex=1",
      "-interaction=nonstopmode",
      "-file-line-error",
      "%DOC%"
    ]
  },
  {
    "name": "pdflatex",
    "command": "pdflatex",
    "args": [
      "-synctex=1",
      "-interaction=nonstopmode",
      "-file-line-error",
      "%DOC%"
    ]
  },
  {
    "name": "pdflatex-with-shell-escape",
    "command": "pdflatex",
    "args": [
      "--shell-escape",
      "-synctex=1",
      "-interaction=nonstopmode",
      "-file-line-error",
      "%DOC%"
    ]
  },
  {
    "name": "latexmk",
    "command": "latexmk",
    "args": [
      "-synctex=1",
      "-interaction=nonstopmode",
      "-file-line-error",
      "-pdf",
      "%DOC%"
    ]
  },
  {
    "name": "bibtex",
    "command": "bibtex",
    "args": [
      "%DOCFILE%"
    ]
  }
],
\end{minted}

\textbf{注意:}虽然左侧插件默认添加了编译方式(\mintinline{shell}{pdflatex} 与 \mintinline{shell}{bibtex}),也必须将其编译方式的设置(比如 \mintinline{shell}{pdflatex} 等)添加到右侧用户设置中。另外上面我们分别为 \mintinline{shell}{pdflatex} 和 \mintinline{shell}{xelatex} 添加了 \mintinline{shell}{--shell-escape} 参数,一个典型的应用场景就是编译包含 \mintinline{tex}{minted} 宏包的文件时。

\subsection{编译组合(\mintinline{shell}{recipe})}
如果我们要对一个文档/项目完整的编译(比如 \mintinline[breaklines]{shell}{pdflatex -> bibtex -> pdflatex -> pdflatex})我们需要用到编译组合(\mintinline{shell}{recipes})。LaTeX Workshop 默认添加了两个 \mintinline{shell}{recipes},分别是 \mintinline{shell}{latexmk} 和 \mintinline[breaklines]{shell}{pdflatex -> bibtex -> pdflatex*2},可以通过点击左侧新增的 \TeX{} 按钮 \includegraphics[width=0.028\textwidth]{tex.png},然后点击 \mintinline{shell}{Build LaTeX project},选择适合的编译组合。


\begin{figure}[htbp]
  \centering
  \includegraphics[width=0.5\textwidth]{compile.png}
  \caption{编译组合的选择}
  \label{fig:example}
\end{figure}


我们之前添加了 \mintinline{shell}{xelatex} 编译方式,我们这里配置下 \mintinline{shell}{xelatex} 的完整编译链 \mintinline[breaklines]{shell}{xelatex -> bibtex -> xelatex*2},另外补充单次编译的 \mintinline{shell}{recipes}。方法和之前类似,打开用户配置文件,将如下 JSON 添加到用户配置中即可。添加新的编译组合之后需要重启 VS Code 才能在 \TeX{} 按钮下看到。


\begin{minted}[frame=single]{json}
"latex-workshop.latex.recipes": [
  {
    "name": "PDFLaTeX",
    "tools": [
      "pdflatex"
    ]
  },
  {
    "name": "PDFLaTeX with Shell Escape",
    "tools": [
      "pdflatex-with-shell-escape"
    ]
  },
  {
    "name": "PDFLaTeX Auto",
    "tools": [
      "pdflatex-latexmk"
    ]
  },
  {
    "name": "XeLaTeX",
    "tools": [
      "xelatex"
    ]
  },
  {
    "name": "XeLaTeX with Shell Escape",
    "tools": [
      "xelatex-with-shell-escape"
    ]
  },
  {
    "name": "XeLaTeX Auto",
    "tools": [
      "xelatex-latexmk"
    ]
  },
  {
    "name": "PDFLaTeX -> BibTeX -> PDFLaTeX*2",
    "tools": [
      "pdflatex",
      "bibtex",
      "pdflatex",
      "pdflatex"
    ]
  },
  {
    "name": "XeLaTeX -> BibTeX -> XeLaTeX*2",
    "tools": [
      "xelatex",
      "bibtex",
      "xelatex",
      "xelatex"
    ]
  },
  {
    "name": "latexmk",
    "tools": [
      "latexmk"
    ]
  },
  {
    "name": "BibTeX",
    "tools": [
      "bibtex"
    ]
  },
],
\end{minted}

\mintinline{shell}{settings} 目录下提供了一个比较完整的配置文件 \href{https://github.com/EthanDeng/vscode-latex/blob/master/settings/settings.json}{settings.json},包括了 \mintinline{tex}{latexmk}\footnote{参考了啸行的《编辑器 texstudio 和 vscode 使用总结》} 与 \mintinline{shell}{pdflatex/xelatex} 以及 \mintinline{shell}{--shell-escape} 的搭配使用。需要的可以下载替换掉用户配置 \mintinline{shell}{tools} 和 \mintinline{shell}{recipe} 部分。另外 \mintinline{shell}{example} 目录下提供一个测试完整编译方式的代码 \href{https://github.com/EthanDeng/vscode-latex/blob/master/example/content.tex}{tex}, \href{https://github.com/EthanDeng/vscode-latex/blob/master/example/info.bib}{bib}, \href{https://github.com/EthanDeng/vscode-latex/blob/master/example/content.pdf}{pdf},你可以用来测试能否编译。效果图如图~\ref{fig:example}:

\begin{figure}[htbp]
  \centering
  \includegraphics[width=0.8\textwidth]{example.png}
  \caption{编译效果图}
  \label{fig:example}
\end{figure}


\subsection{指定编译方式}
在 Sublime Text 或者 \TeX{} Studio 中,可以在文件的首行指定编译方式(\mintinline{shell}{% !TEX program})以及主文档(\mintinline{shell}{% !TEX root}),\LaTeX{} Workshop 也把这个功能添加到了其中,使用方法完全一样。\mintinline{shell}{% !TEX program} 和 \mintinline{shell}{% !TEX root} 被称为 Magic Command(魔法注释)。


%下面这部分可以加到其它编译方式里面。

之前的魔法注释还不支持指定参数,现在已经可以了。

首先可以给它的配置文件里面先加上默认的魔法注释参数,如下:
\begin{minted}[frame=single]{json}
"latex-workshop.latex.magic.args": [
  "-synctex=1",
  "-interaction=nonstopmode",
  "-file-line-error",
  "%DOC%"
]
"latex-workshop.latex.magic.bib.args": [
  "%DOCFILE%"
],
\end{minted}




这时候在主文件开头加上魔法注释\mintinline{shell}{% !TEX program = xelatex}
运行时相当于

\mintinline{shell}{xelatex.exe -synctex=1 -interaction=nonstopmode  -file-line-error "yourfile".tex}。

如果你还要用\mintinline{shell}{bibtex}编译参考文献的话在上一个魔法注释下面再加一个魔法注释:

\mintinline{shell}{% !BIB program = bibtex},如果用的是\mintinline{shell}{biber},后面的参数可以改成\mintinline{shell}{biber}。

但是你可能想让不同的文档编译时可以带不同的参数,这时候还可以再加一个魔法注释:

\mintinline{shell}{% !TEX options = -synctex=1 -interaction=nonstopmode -file-line-error "%DOC%"},等号后面的参数可以自己按需求修改。

看到这里\mintinline{shell}{texify}和\mintinline{shell}{latexmk}用户觉得比较实用了了。不需要修改配置文件,直接用魔法注释就可以实现你想要的编译。

下面来说明这两种的魔法注释怎么写。

对于\mintinline{shell}{texify}:
想用xelatex的话这么配置魔法注释

\begin{minted}[frame=single,tabsize=2,breaklines=true]{json}
% !TEX program = texify
% !TEX options = --synctex --pdf --engine=xetex --tex-option=\"-interaction=nonstopmode\" --tex-option=\"-file-line-error\" "%DOC%.tex"
\end{minted}


只用普通的pdflatex的话用下面这个魔法注释
\begin{minted}[frame=single,tabsize=2,breaklines=true]{json}
% !TEX program = texify
% !TEX options = --synctex --pdf --tex-option=\"-interaction=nonstopmode\" --tex-option=\"-file-line-error\" "%DOC%.tex"
\end{minted}




对于\mintinline{shell}{latexmk}用户也是类似的,但是目前有点奇怪如果你使用了\mintinline{shell}{"latex-workshop.latex.outDir":"%DOC%"},这种指定文件路径的参数,魔法注释明明指定了要生成pdf,可是它最终只编译出dvi,可能\mintinline{shell}{latexmk}用户这个魔法注释还是不能非常愉快的玩耍。所以不要去改变辅助文件的存放位置比较好。

指定用pdflatex编译

\begin{minted}[frame=single,tabsize=2,breaklines]{json}
% !TEX program = latexmk 
% !TEX options = -synctex=1 -interaction=nonstopmode -file-line-error -pdf "%DOC%"
\end{minted}

指定用xelatex编译
\begin{minted}[frame=single]{json}
% !TEX program = latexmk 
% !TEX options = -synctex=1 -interaction=nonstopmode -file-line-error -xelatex "%DOC%"
\end{minted}


下面给一个MWE:

\begin{minted}[frame=single,breaklines]{latex}
% !TEX program = texify
% !TEX options = --synctex --pdf --tex-option=\"-interaction=nonstopmode\" --tex-option=\"-file-line-error\" "%DOC%.tex"
%魔法注释加在上面,替换称你想用的就行了。记得下面两行选择相应的一行使用。
\documentclass[UTF8]{ctexart}%测试pdflatex时用这个,把下面那句注释掉。
%\documentclass{ctexart}%测试xelatex时用这个,把上面那句注释掉。

\title{A Silly Article}
\author{Me}

\begin{document}
\maketitle

My favorite article is \cite{supercurves}. 
我最喜欢的文章是\cite{supercurves}。
\bibliographystyle{plain}
\bibliography{shortbib} %还需要建立一个shortbib.bib文件和这个文件放一起。
\end{document}
\end{minted}

\begin{minted}[frame=single]{latex}
%这是shortbib.bib文件
@ARTICLE{supercurves,
AUTHOR={Minghui Xia},
TITLE={Image Registration by "Super-Curves"},
JOURNAL={IEEE Transactions on Image Processing},
YEAR={1985},
VOLUME = {13},
PAGES={720},
}
\end{minted}

将上述代码保存,然后使用快捷键 \mintinline{shell}{Ctrl+Alt+B},系统会自动选择相应的编译方式。如果没有其他问题,就能正常编译。



\section{其他配置}
\subsection{配置快捷键}
\LaTeX{} Workshop 的快捷键并不友好,我们可以自定义快捷键,方法如下:点击 VS Code 左下角的齿轮(设置),选择 \mintinline{shell}{键盘快捷方式}。


\begin{itemize}
  \item 搜索 \mintinline{shell}{latex build project},将默认的快捷方式改为 \mintinline{shell}{Ctrl+B}
  \item 搜索 \mintinline{shell}{latex build with recipe},将其改为 \mintinline{shell}{Ctrl+R}
  \item 搜索 \mintinline{shell}{latex view pdf file},将其改为 \mintinline{shell}{Ctrl+L}
  \item 你还可以补充其他快捷键。
\end{itemize}

配置好快捷键之后,之后当你指定了编译方式时可以直接使用快捷键 \mintinline{shell}{Ctrl+B} 编译一次文档。当你需要完整编译整个文档(文献,目录等),使用快捷键 \mintinline{shell}{Ctrl+R},选择完整的编译方案即可。是不是方便多了?

\textbf{补充}:为了方便快捷键的绑定,这里直接提供设置文件(\mintinline{shell}{keybindings.json})的 \href{https://github.com/EthanDeng/vscode-latex/blob/master/settings/keybindings.json}{下载},下载之后只需要将其替换用户的快捷键设置即可(默认位置为:\mintinline[breaklines]{shell}{C:/Users/<Your User Name>/AppData/Roaming/Code/User})。


\subsection{配置阅读器以及自动编译}
还有其他几个设置需要提一下,由于笔记本的屏幕很小,我并不习惯使用 VS Code 自带的 PDF 阅读器作为预览的阅读器,可以设置 \mintinline{shell}{SumatraPDF} 作为 PDF 阅读器。另外,自动编译选项我也选择关闭。

\begin{minted}[frame=single]{json}
"latex-workshop.view.pdf.viewer": "external",
"latex-workshop.latex.autoBuild.onSave.enabled": false,
\end{minted}

\subsection{Mac 系统 PDF 阅读器配置}
如何让 VS Code 在 Mac 上与诸如 Skim 的外部 PDF 浏览器配合工作——特别是 \LaTeX{} 的正反跳转,请参考小 L 的 \href{https://liam0205.me/2018/04/24/Working-with-VSCode-on-macOS-configuration-LaTeX-workshop-and-Skim/}{配置说明}。

%如果你认为值得分享的话可以加进去
\section{texify和latexmk部分的配置文件}
如果你想用配置文件的话,下面我提供\mintinline{shell}{texify}和\mintinline{shell}{latexmk}的配置。

\begin{minted}[frame=single]{json}
"latex-workshop.latex.tools": [
{
        "name": "texifyxe",
        "command": "texify",
        "args": [
        "--synctex",
        "--pdf",
        "--engine=xetex",
        "--tex-option=\"-interaction=nonstopmode\"",
        "--tex-option=\"-file-line-error\"",
        "%DOC%.tex"
        ],
        "env": {}
    },
    {
        "name": "texifypdf",
        "command": "texify",
        "args": [
        "--synctex",
        "--pdf",
        "--tex-option=\"-interaction=nonstopmode\"",
        "--tex-option=\"-file-line-error\"",
        "%DOC%.tex"
        ],
        "env": {}
    },
    {
        "name": "latexmkxe",
        "command": "latexmk",
        "args": [
        "-synctex=1",
        "-interaction=nonstopmode",
        "-file-line-error",
        "-xelatex",
        //"-outdir=%OUTDIR%",
        "%DOC%.tex"
        ]
    },
    {
        "name": "latexmkpdf",
        "command": "latexmk",
        "args": [
        "-synctex=1",
        "-interaction=nonstopmode",
        "-file-line-error",
        "-pdf",
        //"-outdir=%OUTDIR%",
        "%DOC%.tex"
        ]
    },
    ],
    
    "latex-workshop.latex.recipes": [
    {
        "name": "texifyxe",
        "tools": ["texifyxe"]
    },
    {
        "name": "texifypdf",
        "tools": ["texifypdf"]
    },
    {
        "name": "latexmkxe",
        "tools": ["latexmkxe"]
    },
    {
        "name": "latexmkpdf",
        "tools": ["latexmkpdf"]
    },
    ],
    
    "latex-workshop.latex.magic.args": [
    "-synctex=1",
    "-interaction=nonstopmode",
    "-file-line-error",
    "%DOC%"
    ],
    "latex-workshop.latex.magic.bib.args": [
    "%DOCFILE%"
    ],
    
    //"latex-workshop.latex.outDir":"%DOC%",
    
\end{minted}



\section*{参考资料}
\begin{itemize}
    \item 本文最早的版本 \href{http://www.latexstudio.net/archives/11087.html}{LaTeX 技巧 916:Visual Studio Code 搭建 LaTeX 编写环境(已被删)}.
    \item 更新版本为 \href{http://www.latexstudio.net/archives/12260.html}{LaTeX技巧932:如何配置Visual Studio Code作为LaTeX编辑器[新版更新]}
    \item \href{https://github.com/James-Yu/LaTeX-Workshop}{Github: LaTeX-Workshop}
    \item \href{https://stackoverflow.com/questions/39775406/how-to-turn-off-matching-highlighting}{How to turn off matching highlighting}
\end{itemize}


\end{document}
